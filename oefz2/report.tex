% !TEX TS-program = pdflatex
% !TEX encoding = UTF-8 Unicode

% This is a simple template for a LaTeX document using the "article" class.
% See "book", "report", "letter" for other types of document.

\documentclass[11pt]{article} % use larger type; default would be 10pt
\usepackage[utf8]{inputenc} % set input encoding (not needed with XeLaTeX)
\usepackage{graphicx}
\usepackage{subcaption}

\usepackage{amsmath}
\usepackage{float}
\usepackage[parfill]{parskip}
\usepackage{ amssymb }

\graphicspath{ {./img/} }

\title{Deterministische en stochastische integratie technieken}

\author{Willem Melis}
%\date{} % Activate to display a given date or no date (if empty),
         % otherwise the current date is printed

\begin{document}
\maketitle
\newpage
%\tableofcontents
\newpage
\section{vraag1}
\section{vraag2}
\section{vraag3}

Zoals aangegeven op bladzijde 240 van de paper uit de cursus zullen symetrische nul regels van een even graad nul zijn. Dit is het geval bij de even functies $f_1$,$f_3$ en $f_4$. Raar genoeg is dit niet zo bij $f_2$, $f_2$ is de enige functie die niet symmetrisch is. Zou dit de reden kunnen zijn?

\begin{figure}[H]
	\centering
	\begin{subfigure}[b]{0.45\textwidth}
		\includegraphics[width=\textwidth]{vraag3_all.png}
		\caption{alle nul regels, log schaal}
	\end{subfigure}
	\begin{subfigure}[b]{0.45\textwidth}
		\includegraphics[width=\textwidth]{vraag3_oneven}
		\caption{oneven nul regels, log schaal}
	\end{subfigure}
\end{figure}

\section{vraag4}
De even nul regels worden uiteraard niet gebruikt aangezien deze nul zijn.
\section{vraag5}
\section{vraag6}
\begin{figure}[H]
	\centering
	\begin{subfigure}[b]{0.45\textwidth}
		\includegraphics[width=\textwidth]{vraag6_smallr.png}
		\caption{$r_i$}
	\end{subfigure}
	\begin{subfigure}[b]{0.45\textwidth}
		\includegraphics[width=\textwidth]{vraag6_R.png}
		\caption{$R_i$}
	\end{subfigure}
\end{figure}

			
\end{document}
